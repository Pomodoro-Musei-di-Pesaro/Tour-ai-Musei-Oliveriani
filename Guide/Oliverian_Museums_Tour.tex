\documentclass[hidelinks,12pt,a4paper]{article}
\usepackage[italian]{babel}
\usepackage[utf8]{inputenc}
\usepackage{fourier} 

% Stop hyphenation
\usepackage[none]{hyphenat}

% Justifying text
\emergencystretch 3em

% Create arrows
\usepackage{textcomp}

% Remove first empty page
\usepackage{atbegshi}
\AtBeginDocument{\AtBeginShipoutNext{\AtBeginShipoutDiscard}}

% License
\usepackage[
type={CC},
modifier={by-nc-sa},
version={4.0},
]{doclicense}

\begin{document}
	\begin{flushleft}
		
		\title{\textbf{Tour al museo e alla biblioteca Oliveriana}}
		\author{Alice Balestieri\\Francesco Rombaldoni}
		\date{}
		
		\maketitle
		% Adjust page counter
		\setcounter{page}{1}
		\newpage
		
		\tableofcontents
		\newpage
		
		\section{Archivi}
		
		\begin{itemize}
			\item Storico Comunale del Comune di Pesaro (1300-1900).
			\item Archivio Famiglia Perticari.
			\item Archivio Famiglia del Marchese Petrucci(botanica).
			\item Donazione e archivio del Prete Molaroni \textrightarrow Stelle di Novilara.
			\item Archivio Famiglia Olivieri e Passeri \textrightarrow \textbf{Affresco Bosco Sacro} e individuazione ipotetica S.Veneranda Pesaro, realizzato da \textbf{Giovanni Andrea Lazzarini}.
			\item Archivio Storico-Fotografico non solo di Pesaro e provincia.
			\item Archivio Giornali (Periodici) e Riviste (Biblioteca Oliveriana).
			\item Archivio delle Storie dei Palazzi di Pesaro \textrightarrow Che viene utilizzato in sinergia con \textbf{l'archivio di stato degli atti ufficiali} per lavori di architettura e restauro dei palazzi storici.
			\item Libri, documenti e Lettere di carattere Pubblico e donazioni o acquisti da Privati ( Biblioteca Oliveriana).
			\item Archivio di Disegni, incisioni e quadri (tra cui un Raffaello in Biblioteca Oliveriana).
			\item Sculture, Reperti di carattere storico-archeologico \textrightarrow Esposti ai musei Oliveriani.			
		\end{itemize}
	
		\section{Storia della Biblioteca Oliveriana}
		Nel 1892 il primo nucleo del museo si trovava dove oggi ha sede il Conservatorio poi fu trasferita nel 800 all'attuale Palazzo Almerici nel 1916 un forte terremoto portò a spostare la sede del Museo Oliveriano al piano terra. La Biblioteca custodisce e conserva e si occupa del restauro di manoscritti rari e della conservazione di fotografie storiche, giornali (che vengono rilegati per la conservazione = carta economica visione della lettura come usa e getta) ed arricchita di lasciti e donazioni volontarie dei cittadini e le collezioni municipali del Comune di Pesaro delle quali è permesso consultare liberamente con la guida degli esperti i volumi della Biblioteca Oliveriana ai cittadini residenti e previa richiesta ai non residenti la collezione appartenente alla Biblioteca quotidianamente e volendo previa richiesta averne una copia anche in formato "e- book" o fotocopie cartacee come voleva lo stesso Olivieri tutt'ora il carattere pubblico della Biblioteca è quindi così ben evidente.
		
		\subsection{Collezione di Olivieri-Passeri}
		Annibale degli Abati Olivieri (1756) tramite Testamento dona le sue collezioni archeologiche e bibliotecarie, pittoriche ed incisorie. Compreso ciò che gli donò l’amico Giovan Battista Passeri (Pittore e biografo del periodo Barocco). Nel 1787 incrementò questo lascito all'Ente Olivieri tramite un nuovo testamento.
		
		\section{Storia del museo Oliveriano}
		Si trovava a 6km da Pesaro e venne scoperta alla fine del 900 i primi scavi furono quelli meno precisi (per via delle conoscenze dell'epoca) ed erano 2 appezzamenti di aree terriere semplici di collina appartenenti al parroco Molaroni era esposta in maniera ammassata (grande quantità di reperti per far vedere la ricchezza della collezione) che non ne permetteva però allo spettatore di soffermarsi su nulla ed era inizialmente senza un’ordine storico preciso e priva di didascalie. I primi scavi regolari vennero condotti da Edoardo Brizio (1892-93) che era al tempo ispettore del Ministero della Pubblica Istruzione e vennero effettuati per trincea nei vari terreni. Successivamente Autostrade per l’Italia ha voluto ampliare le proprie autostrade con una 3° corsia passando proprio sotto la collina di Novilara (2012-2013) e così facendo avvenne il ritrovamento di altre 150 tombe. Durante lo scavo del 2012-2013 sono state intercettate le trincee di Brizio e quindi gli scavi e il ritrovamento di reperti sono stati ampliati annettendo questa nuova zona. \textit{Il periodo} di questi ultimi reperti \textit{Viene definito periodo orizzontale} (VIII-VII sec a.C.) e corrisponde al periodo del ferro, prende questo nome a causa degli influssi culturali del vicino Oriente e della Grecia, infatti l’idea di regalità appartenente a quel periodo deriva dai poemi Omerici. Da questo tipo di ideali vengono infatti caratterizzate le sepolture che troviamo esposte.
		\begin{itemize}
			\item \textbf{Sepolture di donne come principesse.}\\
			Con gioielli, strumenti per filare e tessere (come Penelope), unguenti, oli profumati e resti di vestiti tradizionali.\\
			Le donne potevano vivere fino a cinquanta anni e la loro statura era di circa dieci centimetri più piccola rispetto a quella degli uomini.
			
			\item \textbf{Sepolture di uomini come guerrieri.}\\
			Con lance, scudi, armature, pettorali, spade, sciabole e accette.\\
			Gli uomini morivano attorno tra i venticinque e i trentacinque anni, spesso in guerra. La statura degli uomini era di un metro e settanta (poco di meno della attuale statura media). Le frecce che troviamo sono rare perché era considerata un'arma da "vigliacchi".
			
			\item \textbf{Rare sepolture di bambini.} (Nonostante l'alto tasso di mortalità storica)\\
			\textit{Fine VIII sec. sepolti in tombe con corredo all'interno della necropoli.}\\
			\textit{Vasi miniaturizzati a forma di cavallino,} in segno del potere di una sorta di “aristocrazia dell’epoca (influsso vasi Bolognesi), Set per filare: Pesi da telaio (oggetti cilindrici chiamati rocchetti) coltellini per tagliare il filo, fuseruole (a forma di stella poste all’ estremità del fuso), aghi per cucire e bastoni di legno attorno ai quali veniva avvolta la lana. \textit{Vasi a lamelle e asta reggi lucerna.}
		\end{itemize}
		Nel VII sec compare lo Spunzone (oggi spiedone) = oggetto da banchetto simbolo di agiatezza perché permetteva di offrire carne nelle tavole e veniva visto come gesto eroico perché veniva narrato che gli eroi si offrissero carne tra pari. In questo periodo troviamo infatti anche coltelli per affettare, strumenti per servire in tavola e porzioni di carne di vario bestiame e animali selvatici.\\
		Altri reperti di particolare interesse sono:
		\begin{itemize}
			\item \textbf{Stelle di Novilara} (di discussa autenticità) per via della stele con iscrizioni. appartenente al parroco Molaroni sembrerebbero dei falsi storici realizzati a fine dell’800 di cui per via di questioni linguistiche sono ancora oggetto di discussioni tra i vari storici e archeologi.
			\item \textbf{L'anemoscopio di Boscovich} datato II-IIIsec.d.C. è un oggetto proveniente da Roma utilizzato per l’ individuazione di venti ed astri posto nella sala iniziale come da testamento di Olivieri.
			\item \textbf{Brucia profumi in metallo} (proveniente dall'ambito Emiliano).
			\item \textbf{Fibula in osso e ambra} (proveniente da Verucchio).
			\item \textbf{Vaso a scarpa} (di provenienza Veneta).
			\item \textbf{Ex voti in terracotta.}
			\item \textbf{Elementi decorativi delle domus} (decorazioni marine con delfini e animali sconosciuti simili a serpenti marini).
			\item \textbf{Lastra metallica dell'ordine dei fabbri e amorino} (palazzo Barignani) assieme ad altri epigrafi di mestieri ormai desueti.
			\item \textbf{Sarcofagi} convertiti in vasche o fontane.
		\end{itemize}
		L’ esposizione di questi reperti in maniera più sfoltita e sintetica vuole ricreare l’ideale delle Wunderkammer o Camere delle Meraviglie termine di origine tedesco dal XVI sec. e XVIII sec. (Oggetti scelti perchè preziosi e particolari conservati all'interno di una singola stanza o armadio). Esempio nel libro illustrato \textbf{La stanza delle meraviglie di Brian Selznick} ( illustratore anche del libro di Hugo Cabret).
		
		% ---------- ADD IMAGES HERE ----------
	
		\vspace*{\fill}
		% Print license shield
		\doclicenseThis
	
	\end{flushleft}
\end{document}