\documentclass[hidelinks,12pt,a4paper]{article}
\usepackage[italian]{babel}
\usepackage[utf8]{inputenc}
\usepackage{fourier} 

% Stop hyphenation
\usepackage[none]{hyphenat}

% Justifying text
\emergencystretch 3em

% Enlarge section & subsection
\usepackage{titlesec}
\titleformat*{\section}{\LARGE\bfseries}
\titleformat*{\subsection}{\Large\bfseries}

% Remove first empty page
\usepackage{atbegshi}
\AtBeginDocument{\AtBeginShipoutNext{\AtBeginShipoutDiscard}}

% License
\usepackage[
type={CC},
modifier={by-nc-sa},
version={4.0},
]{doclicense}

\begin{document}
	\begin{flushleft}
		
		%Enlarge text
		\LARGE
		
		\title{\textbf{Tour al Museo e alla Biblioteca Oliveriana}}
		\author{Alice Balestieri\\Francesco Rombaldoni}
		\date{} 
		
		\maketitle
		
		%Adjust page counter
		\setcounter{page}{1}
		\newpage
		\topskip0pt
		\vspace*{\fill}
			Questo documento ha lo scopo di formare gli Operatori del Museo e Biblioteca Oliveriana riguardo ai reperti e collezioni che ospita lo stabile, per poterli spiegare al meglio ai visitatori.
		\vspace*{\fill}
		\newpage
		\tableofcontents
		\newpage
		
		\section{Archivi}
		
		\begin{itemize}
			\item Storico Comunale del Comune di Pesaro (1300-1900).
			\item Archivio Famiglia Perticari.
			\item Archivio Famiglia del Marchese Petrucci(botanica).
			\item Donazione e archivio del Prete Molaroni \textrightarrow Stelle di Novilara.
			\item Archivio Famiglia Olivieri e Passeri \textrightarrow \textbf{Affresco Bosco Sacro} e individuazione ipotetica S.Veneranda Pesaro, realizzato da \textbf{Giovanni Andrea Lazzarini}.
			\item Archivio Storico-Fotografico non solo di Pesaro e provincia.
			\item Archivio Giornali (Periodici) e Riviste (Biblioteca Oliveriana).
			\item Archivio delle Storie dei Palazzi di Pesaro \textrightarrow Che viene utilizzato in sinergia con \textbf{l'archivio di stato degli atti ufficiali} per lavori di architettura e restauro dei palazzi storici.
			\item Libri, documenti e Lettere di carattere Pubblico e donazioni o acquisti da Privati ( Biblioteca Oliveriana).
			\item Archivio di Disegni, incisioni e quadri (tra cui un Raffaello in Biblioteca Oliveriana).
			\item Sculture, Reperti di carattere storico-archeologico \textrightarrow Esposti ai musei Oliveriani.			
		\end{itemize}
		
		\section{La nascita della Collezione Olivieri-Passeri}
		Annibale degli Abati Olivieri (1756) tramite Testamento donò le sue collezioni archeologiche, bibliotecarie, pittoriche ed incisorie alla Biblioteca;compreso ciò che gli donò l’amico Giovan Battista Passeri (Pittore e biografo del periodo Barocco). Successivamente, 1787 incrementò questo lascito all'Ente Olivieri tramite un nuovo testamento.
		
		\subsection{Storia della Biblioteca Oliveriana}
		
		Nel 1892 il primo nucleo del museo si trovava dove oggi ha sede il Conservatorio, poi a causa dell'ampiamento della collezione e la conseguente mancanza di spazio fu trasferito nel 800 all'attuale Palazzo Almerici.
		A causa di un forte terremoto avvenuto nel 1916 la sede del Museo Oliveriano venne spostata al piano terra.
		\\ \textbf{La Biblioteca custodisce, conserva e si occupa del restauro di manoscritti rari, di fotografie storiche e di quotidiani}\\ (che vengono rilegati per la conservazione a causa della carta economica con cui vengono prodotti, per via della visione della loro lettura come usa e getta).
		La Biblioteca è inoltre arricchita da lasciti e donazioni volontarie dei cittadini e dalle collezioni municipali del Comune di Pesaro, delle quali è permesso consultare liberamente con la guida degli esperti i volumi; sia da parte di cittadini residenti a Pesaro e previa richiesta anche da parte dei non residenti.
		È inoltre possibile avere una copia dei testi anche in formato "e- book" o averne delle fotocopie.
		La volontà di Olivieri era che le sue collezioni e il carattere della gestione della Biblioteca fossero a favore dell'arricchimento culturale da parte del pubblico come tutt'ora è visibile da parte di chi la visita.
		
		
		
		\section{Storia del Museo Oliveriano}
		Il terreno dei siti archeologici si trovava a 6km da Pesaro e venne scoperto casualmente alla fine del 900.
		Questi primi scavi furono quelli meno precisi (per via delle scarse conoscenze dell'epoca), ed erano relativi a due appezzamenti di aree terriere collinari appartenenti al Parroco Molaroni.
		
		I primi scavi regolari vennero condotti da Edoardo Brizio (1892-93) che era al tempo ispettore del Ministero della Pubblica Istruzione e vennero effettuati per trincea nei vari terreni.
		Nel((2012-2013) Autostrade per l’Italia ha voluto ampliare i propri tratti stradali con una 3° corsia, passando sotto la collina di Novilara e così facendo avvenne il ritrovamento di altre 150 tombe.
		\\ \textbf{Il periodo di questi ultimi reperti Viene definito periodo orizzontale}\\ (VIII-VII sec a.C.) e corrisponde al periodo del ferro, prende questo nome a causa degli influssi culturali del vicino Oriente e della Grecia; infatti l’idea di regalità appartenente a quel periodo deriva dai poemi Omerici. Da questo tipo di ideali vengono quindi caratterizzate le sepolture che troviamo esposte nella prima sala del museo.
		Proseguendo nella seconda sala troviamo invece inizialmente dei reperti di natura votiva e sepolcrele e infine proseguendo i più recenti reperti di natura quotidiana presenti all'interno delle domus romane. 
		
		Inizialmente l' esposizione dei reperti all'interno del museo avvenne in maniera ammassata e confusionaria (era collocata una grande quantità di reperti per far vedere la ricchezza della collezione) che non ne permetteva però ai visitatori di soffermarsi su nulla ed era inoltre senza un’ordine storico preciso e priva di didascalie. Con il nuovo e rinnovato allestimento si è voluto sopperire a tali problemi con un nuovo allestimento organizzato in in maniera più sfoltita e sintetica in modo tale da ricreare l’ideale delle Wunderkammer o Camere delle Meraviglie termine di origine tedesco dal XVI sec. e XVIII sec. (Oggetti scelti perchè preziosi e particolari conservati all'interno di una singola stanza o armadio).
		Esempio  che possiamo trovare nel libro illustrato \textbf{La Stanza delle meraviglie di Brian Selznick} ( illustratore anche del libro di Hugo Cabret).
		È inoltre possibile ora, attraverso l'apertura di cassetti osservare parte dei reperti più da vicino.
		
		\begin{itemize}
			\item \textbf{Usanza Sepolcrale di rappresentare le donne come Principesse.}\\
			Con gioielli, strumenti per filare e tessere (come Penelope), unguenti, oli profumati e resti di vestiti tradizionali.\\
			Le donne potevano vivere fino a cinquanta anni e la loro statura era di circa dieci centimetri più piccola rispetto a quella degli uomini.
			
			\item \textbf{Usanza sepolcrale di raffigurare gli uomini come guerrieri.}\\
			Con lance, scudi, armature, pettorali, spade, sciabole e accette.\\
			Gli uomini morivano attorno tra i venticinque e i trentacinque anni, spesso in guerra.
			La statura degli uomini era di un metro e settanta (poco di meno della attuale statura media).
			Le frecce che troviamo sono un reperto raro, perché era considerata un'arma da "vigliacchi".
			
			\item \textbf{Rare sepolture di bambini.} (Nonostante l'alto tasso di mortalità storica)\\
			A Fine VIII sec.venivano sepolti in piccole tombe con corredo all'interno della Necropoli che era caratterizzato da:
			\\ \textbf{Vasi miniaturizzati a forma di cavallino, in segno di potere da parte di una sorta di “aristocrazia dell’epoca, (influsso vasi Bolognesi), Set per filare: Pesi da telaio (oggetti cilindrici chiamati rocchetti) coltellini per tagliare il filo, fuseruole (a forma di stella poste all’ estremità del fuso), aghi per cucire e bastoni di legno attorno ai quali veniva avvolta la lana. Vasi a lamelle e un'asta reggi lucerna, probabilmente erano delle mansioni casalinghe che le madri affidavano ai bambini per farli crescere e farsi aiutare nello svolgimento delle mansioni domestiche.}\\
		\end{itemize}
		Solo nel VII sec. compare come reperto lo Spunzone (oggi spiedone) = oggetto da banchetto simbolo di agiatezza perché permetteva di offrire carne nelle tavole e veniva visto come gesto eroico perché veniva narrato che gli eroi si offrissero carne tra pari. In questo periodo troviamo infatti anche coltelli per affettare, strumenti per servire in tavola e porzioni di carne di vario bestiame e animali selvatici.\\
		Altri reperti di particolare interesse sono:
		\begin{itemize}
			\item \textbf{Stelle di Novilara} (di discussa autenticità) per via della stele con iscrizioni, appartenente al parroco Molaroni sembrerebbe essere un falso storico realizzato a fine dell’800 di cui per via di questioni linguistiche è ancora oggetto di discussioni tra i vari storici e archeologi; cosa che la rende tutt'ora una delle principali attrazioni del Museo.
			Salendo le scale della Biblioteca possiamo infatti trovare sulla destra accanto all'ingresso un modellino in legno che riproduce la misteriosa nave che compare sulla Stele.
			\item \textbf{L'anemoscopio di Boscovich} 
			Datato II-IIIsec. d.C. è un oggetto proveniente da Roma utilizzato per l’ individuazione di venti ed astri posto nella sala iniziale come da testamento di Olivieri è anch'esso pezzo assai raro da vedere in un Museo motivo per cui Olivieri voleva dargli risalto.
			\item \textbf{Reperto di vesti con decorazioni di Donna} 
			Ritrovato nelle sepolture, la cui conservazione eccelmente avvenuta ci permette di comprendere i motivi decorativi degli abiti dell'epoca da parte delle famiglie di origini nobiliari.
			\item \textbf{Brucia profumi in metallo} (proveniente dall'ambito Emiliano).
			\item \textbf{Fibula in osso e ambra} (proveniente da Verucchio).
			\item \textbf{Vaso a scarpa} (di provenienza Veneta).
			\item \textbf{Ex voti in terracotta.}
			\item \textbf{Elementi decorativi delle domus} (decorazioni marine con delfini e animali sconosciuti simili a serpenti marini).
			\item \textbf{Lastra metallica dell'ordine dei fabbri e amorino} (palazzo Barignani) assieme ad altri epigrafi di mestieri ormai desueti.
			\item \textbf{Sarcofagi} convertiti in vasche o fontane.
		\end{itemize}
		% ---------- ADD IMAGES HERE ----------
		
		\section{Fonti}
		Appunti tratti dall'incontro con Brunella Paolini, presso i Musei Oliveriani. \\
		% Rilaborazione del documento fornito da Emmanuela Gaudenzi 
		
		\vspace*{\fill}
		% Print license shield
		\doclicenseThis
	\end{flushleft}
\end{document}